%Introductory beamer presentation: quickbeamer2.tex

\documentclass{beamer}
\usetheme{Berkeley}
\usepackage[utf8]{inputenc}
\usepackage[T1]{fontenc}
\usepackage[backend=biber]{biblatex}
\bibliography{thesis.bib}
\setbeamertemplate{bibliography item}{}

\begin{document}
\title[Deep Unfolding of Wirtinger Flow type Schemes]{Deep Unfolding of Wirtinger Flow type Schemes}
\author[]{Ali Darijani\\ \href{mailto:ali.darijani.me@gmail.com}{ali.darijani.me@gmail.com}\newline
\\ Supervisor: Prof. Benjamin Berkels\\ \href{mailto:berkels@aices.rwth-aachen.de}{berkels@aices.rwth-aachen.de}
}
% \author[author1]{author1\inst{1}\\[1ex]  {\small supervisor1\inst{1,2}}}

\institute{RWTH Aachen University} 
\date{\today}

\begin{frame}
\titlepage
\end{frame}

\begin{frame}
\frametitle{Outline}

\tableofcontents[pausesections]
% \tableofcontents
\end{frame}

\section{Phase Problem}

\begin{frame}
\frametitle{Phase Problem}

In this note, we prove the following result:

\begin{theorem} 
There exists an infinite complete distributive 
lattice~$K$ with only the two trivial complete 
congruence relations.
\end{theorem}
\end{frame}

\section[Construction]{The $\Pi^{*}$ construction}

\begin{frame}
\frametitle{The $\Pi^{*}$ construction}

The following construction is crucial in the proof
of our Theorem:

\begin{definition} 
Let $D_{i}$, for $i \in I$, be complete distributive 
lattices satisfying condition~\textup{(J)}.  Their 
$\Pi^{*}$ product is defined as follows:
\[
   \Pi^{*} ( D_{i} \mid i \in I ) = 
   \Pi ( D_{i}^{-} \mid i \in I ) + 1;
\]
that is, $\Pi^{*} ( D_{i} \mid i \in I )$ is 
$\Pi ( D_{i}^{-} \mid i \in I )$ with a new 
unit element. 
\end{definition}
\end{frame}

\begin{frame}
\frametitle{Illustrating the construction}

\centering\includegraphics{products}
\end{frame}

\begin{frame}
\frametitle{Notation}

If $i \in I$ and $d \in D_{i}^{-}$, then
\[
  \langle \ldots, 0, \ldots, d, \ldots, 0, \ldots \rangle
\]
is the element of $\Pi^{*} ( D_{i} \mid i \in I )$ whose 
$i$-th component is $d$ and all the other components 
are $0$.

See also Ernest~T. Moynahan, 1957.
\end{frame}

\section[Second result]{The second result}

\begin{frame}
\frametitle{The second result}

Next we verify the following result:

\begin{theorem}
Let $D_{i}$, $i \in I$, be complete distributive 
lattices satisfying condition~\textup{(J)}.  
Let $\Theta$ be a complete congruence relation on 
$\Pi^{*} ( D_{i} \mid i \in I )$. 
If there exist $i \in I$ and $d \in D_{i}$ with 
$d < 1_{i}$ such that, for all $d \leq c < 1_{i}$, 
\begin{equation*}
   \langle \ldots, d, \ldots, 0, \ldots \rangle \equiv 
   \langle \ldots, c, \ldots, 0, \ldots \rangle 
   \pmod{\Theta}, 
\end{equation*}
then $\Theta = \iota$.
\end{theorem}
\end{frame}

\section{Proof}

\begin{frame}
\frametitle{Starting the proof}

Since 
\begin{equation*}
\langle \ldots, d, \ldots, 0, \ldots \rangle \equiv 
\langle \ldots, c, \ldots, 0, \ldots \rangle 
\pmod{\Theta}, 
\end{equation*}
and $\Theta$ is a complete congruence relation, 
it follows from condition~(J) that
\begin{equation*}
 \langle \ldots, d, \ldots, 0, \ldots \rangle \equiv
 \bigvee ( \langle \ldots, c, \ldots, 0, \ldots \rangle 
 \mid d \leq c < 1 ) \pmod{\Theta}. 
\end{equation*}
\end{frame}

\begin{frame}
\frametitle{Completing the proof}

Let $j \in I$, $j \neq i$, and let $a \in D_{j}^{-}$. 
Meeting both sides of the congruence 
with $\langle \ldots, a, \ldots, 0, \ldots \rangle$, 
we obtain that
\begin{equation*}
   0 = \langle \ldots, a, \ldots, 0, \ldots \rangle 
     \pmod{\Theta}, 
\end{equation*}
Using the completeness of $\Theta$ and the penultimate equation, 
we get:
\[
   0 \equiv \bigvee ( \langle \ldots, a, \ldots, 0, 
     \ldots \rangle \mid a \in D_{j}^{-} ) = 1 
     \pmod{\Theta}, 
\]
hence $\Theta = \iota$.
\end{frame}

\section{References}

\begin{frame}
\frametitle{References}

\begin{thebibliography}{9}

\bibitem{sF90}
Soo-Key Foo, 
\emph{Lattice Constructions}, 
Ph.D. thesis, 
University of Winnebago, Winnebago, MN, December, 1990.

\bibitem{gM68}
George~A. Menuhin, 
\emph{Universal Algebra}, 
D.~van Nostrand, Princeton, 1968.

\bibitem{eM57}
Ernest~T. Moynahan, 
\emph{On a problem of M. Stone},
Acta Math. Acad. Sci. Hungar. \textbf{8} (1957), 
455--460.

\bibitem{eM57a}
Ernest~T. Moynahan, 
\emph{Ideals and congruence relations in lattices.} II,
Magyar Tud. Akad. Mat. Fiz. Oszt. K\"{o}zl. \textbf{9} 
(1957), 417--434.

\end{thebibliography}
\end{frame}


% \begin{frame}
%    \frametitle{References}
%    % This prints the bibliography on the slide
%    \nocite{bib:book-convex-boyd}
%    \nocite{bib:paper-wf-candes}
%    \printbibliography
%    \end{frame}
\end{document}

