%%%%%%%%%%%%%%%%%%%%%%%%%%%%%%%%%%%%%%%%%%%%%%%%%%%%%%%%%%%%%%%%%%%%%%%
%%%%%%%%%%%%%%%%%%%%%%%%%%%% Deep Unfolding %%%%%%%%%%%%%%%%%%%%%%%%%%%
%%%%%%%%%%%%%%%%%%%%%%%%%%%%%%%%%%%%%%%%%%%%%%%%%%%%%%%%%%%%%%%%%%%%%%%

\begin{frame}
  \frametitle{Review of Deep Learning Pros}
  \begin{itemize}
    \pause
    \item can extract extremely complicated mappings
    \pause
    \item requires little to no domain knowledge
    \pause
    \item forth runner in some analytically intractable problems
  \end{itemize}
\end{frame}

\begin{frame}
  \frametitle{Review of Deep Learning Cons}
  \begin{itemize}
    \pause
    \item requires large and high quality datasets
    \pause
    \item requires tremendous raw computational power
    \pause
    \item there is no interpretability most of the times
  \end{itemize}
\end{frame}

\begin{frame}
  \frametitle{Deep Unfolding}
  \pause
  Deep Unfolding is a new approach in the explainable/model-based Artificial Intelligence 
  that is geared towards the iterative solutions.
\end{frame}

\begin{frame}
  \frametitle{Deep Unfolding(Schematic Diagram)}
  \begin{figure}
    \centering
    % \captionsetup{justification=centering}
    \resizebox{1.0\textwidth}{!}{\input{../document/tikz/diagrams/unfolding.tex}}
    % \caption{The Schematic Unfolding of an Iterative Algorithm}
    \label{fig:deep_unfolding}
  \end{figure}
\end{frame}

\begin{frame}
  \frametitle{Deep Unfolding}
  \begin{itemize}
    \pause
    \item interpreting an iteration in the form of a mapping between two layers. 
    \pause
    \item nonlinearities are based on domain knowledge.
    \pause
    \item parameters have meanings.
    \pause
    \item smaller datasets and the perks that come with it.
    \pause
    \item easier initialization of the parameters.
  \end{itemize}
\end{frame}
