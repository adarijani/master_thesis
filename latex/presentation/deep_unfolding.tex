%%%%%%%%%%%%%%%%%%%%%%%%%%%%%%%%%%%%%%%%%%%%%%%%%%%%%%%%%%%%%%%%%%%%%%%
%%%%%%%%%%%%%%%%%%%%%%%%%%%% Deep Unfolding %%%%%%%%%%%%%%%%%%%%%%%%%%%
%%%%%%%%%%%%%%%%%%%%%%%%%%%%%%%%%%%%%%%%%%%%%%%%%%%%%%%%%%%%%%%%%%%%%%%

\begin{frame}
  \frametitle{Review of Deep Learning Pros}
  \begin{itemize}
    \pause
    \item can extract extremely complicated mappings or extremely subtle features
    \pause
    \item requires little to no knowledge about the exact internals of the problem
    \pause
    \item currently can beat human level performance in lots of areas; object 
    detection (single and multi), object tracking(single and multi), anomaly detection, fraud detection, and strategical games just to name a few.
  \end{itemize}
\end{frame}

\begin{frame}
  \frametitle{Review of Deep Learning Cons}
  \begin{itemize}
    \pause
    \item requires large and high quality datasets like the ImageNet\cite{SVLL}, in the case of image classification, which are expensive to acquire and store,
    \pause
    \item requires tremendous raw computational power at the level of top 500 supercomputers and storage which in turn would result in large electricity bills and expensive maintenance costs,
    \pause
    \item since mostly there is no interpretability associated with models, there will be no reasoning when a model gives objectively wrong answers.
  \end{itemize}
\end{frame}

\begin{frame}
  \frametitle{Deep Unfolding}
  \pause
  Deep Unfolding is a new approach in the explainable/model-based Artificial Intelligence 
  that geared towards the iterative solutions.
\end{frame}

\begin{frame}
  \frametitle{Deep Unfolding(Schematic Diagram)}
  \begin{figure}
    \centering
    % \captionsetup{justification=centering}
    \resizebox{1.0\textwidth}{!}{\input{../document/tikz/diagrams/unfolding.tex}}
    % \caption{The Schematic Unfolding of an Iterative Algorithm}
    \label{fig:deep_unfolding}
  \end{figure}
\end{frame}

\begin{frame}
  \frametitle{Deep Unfolding}
  \begin{itemize}
    \pause
    \item We interpret an iteration in the form of a mapping between two layers. 
    \pause
    \item Domain knowledge is required otherwise the nonlinearity associated with the layer is most likely can not be found by luck.
    \pause
    \item In \nns we add parameters mostly haphazardly while in \du we do it based on analytical arguments.
    \pause
    \item Compared to general \nns we have smaller datasets and the perks that come with it.
    \pause
    \item Initialization of the weight matrices and biases are easier due to domain knowledge in the training.
  \end{itemize}
\end{frame}
