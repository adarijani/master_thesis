\chapter{Mathematical Preliminaries}



\begin{Def}\label{def:dft}
    \emph{$2$D Discrete Fourier Transform}\\
    The \emph{$2$D Discrete Fourier Transform} of the $2$D array $\boldsymbol{X} \in \mathbb{C}^{N \times M}$ is denoted by 
    $\hat {\boldsymbol{X}} \in \mathbb{C}^{N \times M}$ and is defined by
    \begin{equation}\label{eq:dft}
        \{\hat {\boldsymbol{X}}\}_{k,l} \coloneq \frac{1}{MN}\sum_{m=0}^{M-1}\sum_{n=0}^{N-1} \{{\boldsymbol{X}}\}_{n,m}\exp\left({\frac{-2\pi ink}{N}}\right)\exp\left({\frac{-2\pi iml}{M}}\right)
    \end{equation}
    and the get back the original array one can use the inversion formula
    \begin{equation}\label{eq:idft}
        \{{\boldsymbol{X}}\}_{n,m} \coloneq \sum_{k=0}^{N-1}\sum_{l=0}^{M-1} \{\hat {\boldsymbol{X}}\}_{k,l}\exp\left({\frac{2\pi ink}{N}}\right)\exp\left({\frac{2\pi iml}{M}}\right).
    \end{equation}    
    
\end{Def}


\begin{Thm}\label{theorem:dft is unitary}
    
    Here goes the actual theorem description.
\end{Thm}