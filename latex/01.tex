\chapter{Introduction}
\label{chap:Introduction}




\section{Related work}
\begin{itemize}
\item Citations can be done with \texttt{biblatex}. Here is how to cite a book \cite{GoWo07}, an article \cite{Ar66}, a proceedings paper \cite{Ch05} and a technical report \cite{ChEsNi04}. In this example, the actual references are defined in the database file \texttt{thesis.bib}, which is included in the header of this document.
\item References should be done using \texttt{\textbackslash cref}. This way, the type of the object we are referencing is automatically added. For instance, \enquote{\texttt{\textbackslash cref\{chap:Introduction\}}} leads to \enquote{\cref{chap:Introduction}}.
\item Plots, even from data files, can be done with TikZ, cf. \cref{fig:TikzExample}
\item Values with units should preferably printed with the \texttt{siunitx} package, e.g. \SI{1}{\metre} is the result of \enquote{\texttt{\textbackslash SI\{1\}\{\textbackslash metre\}}}. This works both in text and in math mode.
\end{itemize}
$\gls{absolute value}$
\begin{figure}[h]
\centering
\begin{tikzpicture}
    \begin{axis}[width = 0.8\textwidth, height = 25em,
                ymode = log, scaled x ticks = false, tick align = outside, ymajorgrids, xmajorgrids, tick pos = left,
                xmin = 0, xmax = 900, xtick = {20, 160, 310, 750},
                ymin = 1e-17,
                ymax = 100,
                % ytick = {1e2,1e1,1e0,1e-1,1e-2,1e-3,1e-4,1e-5,1e-6,1e-7,1e-8,1e-9,1e-10,1e-11,1e-12,1e-13,1e-14,1e-15,1e-16,1e-17},
                xlabel=number of CG iterations,
                legend style={fill=none},
                no markers,
                line width=0.8pt
                % every axis plot/.append style={thick}% semithick, thick, very thick,ultra thick
                ]
                
    \addplot[mark=none, color=red] table [x index = {0}] {wf_err.dat};
    \addlegendentry{WF}
    \addplot[mark=none, color=orange] table [x index = {0}] {twf_err.dat};
    \addlegendentry{TWF}
    \addplot[mark=none, color=purple] table [x index = {0}] {rwf_err.dat};
    \addlegendentry{RWF}
    \addplot[mark=none, color=cyan] table [x index = {0}] {irwf_err_1.dat};
    \addlegendentry{IRWF}
    \addplot[mark=none, color=blue] table [x index = {0}] {irwf_err_64.dat};
    \addlegendentry{IMRWF}
    % legend style={fill=none}
    \end{axis}
\end{tikzpicture}
\caption{TikZ can create beautiful plots directly from data files. These plots use vector graphics and their fonts are fully consistent with the fonts of the document.}
\label{fig:TikzExample}
\end{figure}

sdf
fsarg
The style defines multiple mathematical environments. All the environments allow to specify a name as optional parameter, as exemplified in \cref{thm:ExampleTheroem}.
\begin{Thm}[Theorem Name, optional]
\label{thm:ExampleTheroem}
Here goes the actual theorem description.
\end{Thm}
\begin{Proof}
Here goes the proof of the theorem. This environment automatically puts a QED square at its end. Sometimes, the automatic placement is not optimal. In this case, \texttt{\textbackslash qedhere} allows to place the symbol at a specific position, for instance in an equation:
\[a^2+b^2=c^2\qedhere\]
\end{Proof}
\begin{Exp}Example of an example.
\end{Exp}
\begin{Def}This is a definition.
\end{Def}
\begin{Prop}This is a proposition.
\end{Prop}
Equivalence proofs where each direction is shown separately can be formatted using the \texttt{\textbackslash itemize} environment with custom labels. If the proof starts with this environment, put a \texttt{\textbackslash leavevmode} before the environment to ensure that the first direction starts on a new line.
\begin{Proof}\leavevmode
\begin{itemize}[beginpenalty=10000,leftmargin=7ex]
\item[\enquote{$\Rightarrow$}:] First direction.
\item[\enquote{$\Leftarrow$}:] Second direction.\qedhere
\end{itemize}
\end{Proof}

\begin{Lem}This is a lemma.
\end{Lem}
\begin{Cor}This is a corollary.
\end{Cor}

\begin{procedure}
    sdf
\end{procedure}
\begin{problem}
    
\end{problem}