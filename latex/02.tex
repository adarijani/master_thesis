\chapter{Wirtinger Flow}

The whole thing about \emph{Wirtinger Flow} variants started with the seminal work of Candes and Soltanolkotabi\cite{bib:wf}.
The most important improvements chronologically were done by Candes and Chen\cite{bib:twf}, Kolte and Özgür\cite{bib:itwf}, and Zhang et al.\cite{bib:rfw-irwf}.
For a quite extensive survey on \emph{Wirtinger Flow} variants please refer to Liu et al.\cite{bib:wf-survey}. Chandra et al.\cite{bib:phasepack} 
gathered quite number of \emph{Phase Retrieval} methods including a couple of \emph{Wirtinger Flow} variants in the MATLAB\textregistered\space 
problem solving environment in a uniform manner.\\
We quickly go over the problem formulation, difficulties, algorithms, and at the of the chapter we give some numerical experiments we are going
to refer to in the subsequent chapters.

\section{Problem Formulation}
Consider the ray $\boldsymbol{x} \in \mathbb{C}^{n \times 1}$ is emitted onto the object of interest and the diffracted rays are measured as 
$\boldsymbol{y} \in \mathbb{R}^{m \times 1}$ and is connected to the original ray by $\boldsymbol{y} = \varphi(\boldsymbol{A}\boldsymbol{x})$,
where $\boldsymbol{A} \in \mathbb{C}^{m \times n}$ and $\varphi$ the usual element-wise absolute value(or the squared absolute value) from 
$\mathbb{C}^{m \times 1}$ to $\mathbb{R}^{m \times 1}$.\\
Candes and Soltanolkotabi\cite{bib:wf} considered $\varphi$ to be squared element-wise absolute value and the loss function to be quadratic. 
The summary for all the variants in terms of formulation is in table\ref{tab:formulation}  


\begin{table}
	\centering
	\begin{tabular}{||c l c||} 
	 \hline
	 \emph{Wirtinger Flow} Variant 			& $\varphi$ 						& loss functions\\ [0.5ex] 
	 \hline\hline
	 Wirtinger Flow 			 			& $\left|\boldsymbol{z}\right|^2$ 	& quadratic 	\\ 
	 Truncated Wirtinger Flow   			& $\left|\boldsymbol{z}\right|^2$ 	& quadratic 	\\
	 Incrementally Truncated Wirtinger Flow & $\left|\boldsymbol{z}\right|^2$  	& quadratic 	\\
	 Reshaped Wirtinger Flow 				& $\left|\boldsymbol{z}\right|$ 	& quadratic 	\\
	 Incrementally Reshaped Flow 			& $\left|\boldsymbol{z}\right|$ 	& quadratic 	\\ [1ex] 
	 \hline
	\end{tabular}
	\caption{$\varphi$ and the loss function used in \cite{bib:wf}, \cite{bib:twf}, \cite{bib:itwf}, \cite{bib:rfw-irwf}}
	\label{tab:formulation}
	\end{table}
\section{Difficulties}

The loss function is non-convex. Two nice figures can be found in \cite{bib:rfw-irwf} that show the effect of different formulation in the curvature 
but at the same time the non-convexness of the presented problem. For the thorough geometric analysis of the \emph{Phase Retrieval} problem 
we refer to Wright et al.\cite{bib:phase-geometric}   
% \begin{equation} \label{prob:mainproblem}
% 	Recover $\boldsymbol{x} \in \mathbb{R}^n/\mathbb{C}^n$ from measurements $y_i$ given by
% 	\begin{flalign}
% 		y_i=\left|\langle \boldsymbol{a}_i,\mathbf{x}\rangle\right|, \quad \text{for }\; i=1,\cdots,m, \label{eq:mainproblem}
% 	\end{flalign}
% 	where $\boldsymbol{a}_i \in \mathbb{R}^n/\mathbb{C}^n$ are random design vectors (known). 
% \end{equation}















% \begin{itemize}
% \item Wirtinger FLow suggested by \cite{Candes_2015}
% \item Truncated Wirtinger Flow suggested by \cite{chen2016solving}
% \item Incrementally Truncated Wirtinger Flow suggested by \cite{kolte2016phase}
% \item Reshaped Wirtinger Flow and Incrementally Reshaped Wirtinger FLow suggested by \cite{zhang2016reshaped}

% \end{itemize}



















	\begin{algorithm}
	\caption{Reshaped Wirtinger Flow suggested by \cite{zhang2016reshaped}}\label{alg:rwf}
		\textbf{Input}: $\boldsymbol{y}=\{y_i\}_{i=1}^m$, $\{\boldsymbol{a}_i\}_{i=1}^m$; \\
		\textbf{Parameters:}  Lower and upper thresholds $\alpha_l,\alpha_u$ for  truncation in initialization, step size $\mu$;\\
		\textbf{Initialization}: Let $\boldsymbol{z}^{(0)}=\lambda_0 \tilde{\boldsymbol{z}}$, where $\lambda_0=\frac{mn}{\sum_{i=1}^m \|\boldsymbol{a}_i\|_1}\cdot \left(\frac{1}{m}\sum_{i=1}^m y_i\right)$ and $\tilde{\boldsymbol{z}}$ is the leading eigenvector of
		\begin{equation}\label{eq:init_TRWF}
			\boldsymbol{Y} \coloneqq \frac{1}{m}\sum_{i=1}^m y_i\boldsymbol{a}_i \boldsymbol{a}_i^*\boldsymbol{1}_{\{\alpha_l \lambda_0<y_i< \alpha_u \lambda_0\}}.
		\end{equation}
		\textbf{Gradient loop}: for $t=0:T-1$ do
		\begin{flalign}\label{eq:loop_FWF}
			\boldsymbol{z}^{(t+1)}=\boldsymbol{z}^{(t)}- \frac{\mu}{m}\sum_{i=1}^{m}\left(\boldsymbol{a}_i^*\boldsymbol{z}^{(t)}-y_i\cdot\frac{\boldsymbol{a}_i^*\boldsymbol{z}^{(t)}}{|\boldsymbol{a}_i^*\boldsymbol{z}^{(t)}|} \right) \boldsymbol{a}_i.
		\end{flalign}
		\textbf{Output} $\boldsymbol{z}^{(T)}$.
	\end{algorithm}

	\begin{algorithm}[th]
		\caption{Incremental Reshaped Wirtinger Flow (IRWF) suggested by \cite{zhang2016reshaped}}\label{alg:irwf}
		
		\textbf{Input}: $\boldsymbol{y}=\{y_i\}_{i=1}^m$, $\{\boldsymbol{a}_i\}_{i=1}^m$; \\
		\textbf{Initialization}: Same as in RWF (Algorithm \ref{alg:rwf}); \\
		\textbf{Parameters:}  Lower and upper thresholds $\alpha_l,\alpha_u$ for  truncation in initialization, step size $\mu$;\\
		% \textbf{Initialization}: Let $\boldsymbol{z}^{(0)}=\lambda_0 \tilde{\boldsymbol{z}}$, where $\lambda_0=\frac{mn}{\sum_{i=1}^m \|\boldsymbol{a}_i\|_1}\cdot \left(\frac{1}{m}\sum_{i=1}^m y_i\right)$ and $\tilde{\boldsymbol{z}}$ is the leading eigenvector of
		%\begin{equation*}
		%\bY:=\frac{1}{m}\sum_{i=1}^m y_i\ba_i \boldsymbol{a}_i^*\bone_{\{\alpha_l \lambda_0<y_i< \alpha_u \lambda_0\}}.
		%\end{equation*}
		
		 \textbf{Gradient loop}: for $t=0:T-1$ do\\
		 Choose $i_t$ uniformly at random from $\{1,2,\ldots, m\}$, and let
		  \begin{flalign}
				\boldsymbol{z}^{(t+1)}=\boldsymbol{z}^{(t)}- \mu\left(\boldsymbol{a}_{i_t}^*\boldsymbol{z}^{(t)}-y_{i_t}\cdot\frac{\boldsymbol{a}_{i_t}^*\boldsymbol{z}^{(t)}}{|\boldsymbol{a}_{i_t}^*\boldsymbol{z}^{(t)}|} \right) \boldsymbol{a}_{i_t}, \label{eq:incrementalupdate}
		\end{flalign}
		\textbf{Output} $\boldsymbol{z}^{(T)}$.
		\end{algorithm}



		\begin{algorithm}[th]
			\caption{Minibatch Incremetnal Reshaped Wirtinger Flow (minibatch IRWF) suggested by \cite{zhang2016reshaped}}\label{alg:mbirwf}
			
			\textbf{Input}: $\boldsymbol{y}=\{y_i\}_{i=1}^m$, $\{\boldsymbol{a}_i\}_{i=1}^m$; \\
			\textbf{Initialization}: Same as in RWF (Algorithm \ref{alg:rwf}); \\
			%\textbf{Parameters:}  Lower and upper thresholds $\alpha_l,\alpha_u$ for  truncation in initialization, step size $\mu$;\\
			%\textbf{Initialization}: Let $\bz^{(0)}=\lambda_0 \tilde{\bz}$, where $\lambda_0=\frac{mn}{\sum_{i=1}^m \|\ba_i\|_1}\cdot \left(\frac{1}{m}\sum_{i=1}^m y_i\right)$ and $\tilde{\bz}$ is the leading eigenvector of
			%\begin{equation*}%\label{eq:init_TRWF}
			%\bY:=\frac{1}{m}\sum_{i=1}^m y_i\ba_i \boldsymbol{a}_i^*\bone_{\{\alpha_l \lambda_0<y_i< \alpha_u \lambda_0\}}.
			%\end{equation*}
			
			 \textbf{Gradient loop}: for $t=0:T-1$ do\\
			 Choose $\Gamma_t$ uniformly at random from the subsets of  $\{1,2,\ldots, m\}$ with cardinality $k$, and let
			 \begin{flalign}
				\boldsymbol{z}^{(t+1)}=\boldsymbol{z}^{(t)}-\mu\cdot\boldsymbol{A}_{\Gamma_t}^* \left(\boldsymbol{A}_{\Gamma_t}\boldsymbol{z}^{(t)}-\boldsymbol{y}_{\Gamma_t}\odot\text{Ph} (\boldsymbol{A}_{\Gamma_t}\boldsymbol{z}^{(t)})\right), \label{eq:mbirwfUpdate}
			\end{flalign}
			where $\boldsymbol{A}_{\Gamma_t}$ is a matrix stacking $\boldsymbol{a}_i^*$ for $i\in \Gamma_t$ as its rows,  $\boldsymbol{y}_{\Gamma_t}$ is a vector stacking $y_i$ for $i\in \Gamma_t$ as its elements, $\odot$ denotes element-wise product, and $\text{Ph}(\boldsymbol{z})$ denotes a phase vector of $\boldsymbol{z}$.% (stacking phase of each element of $\bz$ together).
			
			\textbf{Output} $\boldsymbol{z}^{(T)}$.
			\end{algorithm}


			$ $