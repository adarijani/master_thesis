\chapter{Mathematical Preliminaries}






\begin{Def}\label{def:1ddft}
    \emph{$1$D Discrete Fourier Transform}\\
    The \emph{$1$D Discrete Fourier Transform} of the $1$D array $\boldsymbol{X} \in \mathbb{C}^{N}$ is denoted by 
    $\hat {\boldsymbol{X}} \in \mathbb{C}^{N}$ and is defined by
    \begin{equation}\label{eq:1ddft}
        \{\hat {\boldsymbol{X}}\}_{k} \coloneqq \frac{1}{N}\sum_{n=0}^{N-1} \{{\boldsymbol{X}}\}_{n}\exp\left({\frac{-2\pi ink}{N}}\right)
    \end{equation}
    and to get back the original array one can use the inversion formula
    \begin{equation}\label{eq:1didft}
        \{{\boldsymbol{X}}\}_{n} \coloneqq \sum_{k=0}^{N-1}\{\hat {\boldsymbol{X}}\}_{k}\exp\left({\frac{2\pi ink}{N}}\right)
    \end{equation}    
\end{Def}

As it is evident from the formula the $1$D Discrete Fourier Transform is a linear transformation therefor 
there a corresponding matrix and basis vectors. The matrix is dense matrix and due to computational efficiency 
is almost never computed directly, however taking a closer look at the basis vectors would shed some light on 
the nature of the said transform and is a time well spent.

\begin{Prop}
    For complex valued vectors $\boldsymbol{x},\boldsymbol{y} \in \mathbb{C}^n$ the following is a proper scalar product
    \begin{equation*}
        \boldsymbol{x} = \left\{x_i\right\}_{i=1,\ldots,n-1}, \quad \boldsymbol{y} = \left\{y_i\right\}_{i=1,\ldots,n-1}
    \end{equation*}
    \begin{equation*}
        \langle\boldsymbol{x},\boldsymbol{y}\rangle \coloneqq \sum_{i=0}^{i=n-1} x_i y_i
    \end{equation*}
\end{Prop}

\begin{proof}
    You can consult a standard book on functional analysis like \cite{rudin} 
\end{proof}






\begin{Prop}\label{Prop:1ddftbasisvectors}
    The basis vectors
    \begin{equation}\label{eq:1ddftbasisvectors}
        \boldsymbol{g}^n = \left\{\exp\left({\frac{-2\pi ink}{N}}\right)\right\}_{k=0,\ldots,N-1}
    \end{equation}
    are orthogonal to each other with respect to the usual inner product for complex valued vectors 
    with the normalization constant of $N$
    \begin{equation}
        \langle\boldsymbol{g}^n,\boldsymbol{g}^{n'}\rangle= N \delta_{n,n'}
    \end{equation}
\end{Prop}

\begin{proof}
    \begin{equation*}
        \boldsymbol{g}^n = \left\{\exp\left({\frac{-2\pi ink}{N}}\right)\right\}_{k=0,\ldots,N-1}, \quad \boldsymbol{g}^{n'} = \left\{\exp\left({\frac{-2\pi in'k}{N}}\right)\right\}_{k=0,\ldots,N-1}
    \end{equation*}
    \begin{equation*}
    \begin{split} 
        \langle\boldsymbol{g}^n,\boldsymbol{g}^{n'}\rangle &= \sum_{k=0}^{N-1} \exp\left({\frac{-2\pi ink}{N}}\right)\overline{\exp\left({\frac{-2\pi in'k}{N}}\right)}
        = \sum_{k=0}^{N-1} \exp\left({\frac{-2\pi ink}{N}}\right)\exp\left({\frac{+2\pi in'k}{N}}\right)\\
        &= \sum_{k=0}^{N-1} \exp\left({\frac{-2\pi i(n'-n)k}{N}}\right)=
        \begin{cases}
            N & \text{when $n = n'$}\text{(trivial)},\\
            0 & \text{when $n\neq n'$}\text{(using geometric sum formula)}.
        \end{cases}
    \end{split}
\end{equation*}
    
\end{proof}








\begin{Def}\label{def:2ddft}
    \emph{$2$D Discrete Fourier Transform}\\
    The \emph{$2$D Discrete Fourier Transform} of the $2$D array $\boldsymbol{X} \in \mathbb{C}^{N \times M}$ is denoted by 
    $\hat {\boldsymbol{X}} \in \mathbb{C}^{N \times M}$ and is defined by
    \begin{equation}\label{eq:2ddft}
        \{\hat {\boldsymbol{X}}\}_{k,l} \coloneq \frac{1}{MN}\sum_{m=0}^{M-1}\sum_{n=0}^{N-1} \{{\boldsymbol{X}}\}_{n,m}\exp\left({\frac{-2\pi ink}{N}}\right)\exp\left({\frac{-2\pi iml}{M}}\right)
    \end{equation}
    and to get back the original array one can use the inversion formula
    \begin{equation}\label{eq:2didft}
        \{{\boldsymbol{X}}\}_{n,m} \coloneq \sum_{k=0}^{N-1}\sum_{l=0}^{M-1}\{\hat {\boldsymbol{X}}\}_{k,l}\exp\left({\frac{2\pi ink}{N}}\right)\exp\left({\frac{2\pi iml}{M}}\right)
    \end{equation}    
\end{Def}

As it is evident from the formula the $2$D Discrete Fourier Transform is a linear transformation therefor 
there a corresponding matrix and basis vectors. The matrix is dense matrix and due to computational efficiency 
is almost never computed directly, however taking a closer look at the basis vectors would shed some light on 
the nature of the said transform and is a time well spent.

\begin{Prop}\label{Prop:2ddftbasisvectors}
    The basis vectors
    \begin{equation}\label{eq:2ddftbasisvectors}
        \boldsymbol{g}^{n,m} = \left\{\exp\left({\frac{-2\pi ink}{N}}\right)\exp\left({\frac{-2\pi iml}{M}}\right)\right\}_{\substack{k=0,\ldots,N-1\\l=0,\ldots,M-1}}
    \end{equation}
    are orthogonal to each other with respect to the usual inner product for complex valued vectors 
    with the normalization constant of $MN$
    \begin{equation}
        \langle\boldsymbol{g}^{n,m},\boldsymbol{g}^{n',m'}\rangle= MN \delta_{n,n'}\delta_{m,m'}
    \end{equation}
\end{Prop}

\begin{proof}
    \begin{align*} 
        \boldsymbol{g}^{n,m}    &= \left\{\exp\left({\frac{-2\pi ink}{N}}\right)\exp\left({\frac{-2\pi iml}{M}}\right)\right\}_{\substack{k=0,\ldots,N-1\\l=0,\ldots,M-1}}\\
        \boldsymbol{g}^{n',m'}  &= \left\{\exp\left({\frac{-2\pi in'k}{N}}\right)\exp\left({\frac{-2\pi im'l}{M}}\right)\right\}_{\substack{k=0,\ldots,N-1\\l=0,\ldots,M-1}}
    \end{align*}
    \begin{equation*}
        \begin{split} 
            \langle\boldsymbol{g}^{n,m},\boldsymbol{g}^{n',m'}\rangle &= \sum_{l=0}^{M-1}\sum_{k=0}^{N-1} \exp\left({\frac{-2\pi ink}{N}}\right)\exp\left({\frac{-2\pi iml}{M}}\right)\overline{\exp\left({\frac{-2\pi in'k}{N}}\right)\exp\left({\frac{-2\pi im'l}{M}}\right)}\\
            &= \sum_{l=0}^{M-1}\sum_{k=0}^{N-1} \exp\left({\frac{-2\pi ink}{N}}\right)\exp\left({\frac{-2\pi iml}{M}}\right)\exp\left({\frac{+2\pi in'k}{N}}\right)\exp\left({\frac{+2\pi im'l}{M}}\right)\\
            &= \sum_{l=0}^{M-1}\sum_{k=0}^{N-1} \exp\left({\frac{-2\pi i(n'-n)k}{N}}\right)\exp\left({\frac{-2\pi i(m'-m)l}{M}}\right)\\
            &= \sum_{k=0}^{N-1} \exp\left({\frac{-2\pi i(n'-n)k}{N}}\right)\sum_{l=0}^{M-1} \exp\left({\frac{-2\pi i(m'-m)k}{M}}\right)\\
            &= 
            \begin{cases}
                MN & \text{when $n = n' \wedge m=m'$}\text{(trivial)},\\
                0 & \text{when $\neg(n = n' \wedge m=m')$}\text{(using geometric sum formula)}.
            \end{cases}    
        \end{split}
    \end{equation*}
    

    
\end{proof}







\begin{Thm}\label{theorem:dft is unitary}
    
    Here goes the actual theorem description.
\end{Thm}


\cite{analysis_tao_II}




