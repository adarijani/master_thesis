\chapter{Wirtinger Flow}

The whole thing about \emph{Wirtinger Flow} variants started with the seminal work of Candes and Soltanolkotabi\cite{Candes2014}.
The most important improvements chronologically were done by Candes and Chen\cite{Chen2015}, Kolte and Özgür\cite{Kolte2016}, and Zhang et al.\cite{Zhang2016}.
For a quite extensive survey on \emph{Wirtinger Flow} variants please refer to Liu et al.\cite{Liu2019}. Chandra et al.\cite{Chandra2017} 
gathered quite number of \emph{Phase Retrieval} methods including a couple of \emph{Wirtinger Flow} variants in the MATLAB\textregistered\space 
problem solving environment in a uniform manner.\\
We quickly go over the problem formulation, difficulties, algorithms, and at the of the chapter we give some numerical experiments we are going
to refer to in the subsequent chapters.

\section{Problem Formulation}
Consider the ray $\boldsymbol{x} \in \mathbb{C}^{n \times 1}$ is emitted onto the object of interest and the diffracted rays are measured as 
$\boldsymbol{y} \in \mathbb{R}^{m \times 1}$ and is connected to the original ray by $\boldsymbol{y} = \varphi(\boldsymbol{A}\boldsymbol{x})$,
where $\boldsymbol{A} \in \mathbb{C}^{m \times n}$ and $\varphi$ the usual element-wise absolute value(or the squared absolute value) from 
$\mathbb{C}^{m \times 1}$ to $\mathbb{R}^{m \times 1}$.\\
Candes and Soltanolkotabi\cite{Candes2014} considered $\varphi$ to be squared element-wise absolute value and the loss function to be quadratic. 
The summary for all the variants in terms of formulation is in table\ref{tab:formulation}  


\begin{table}
	\centering
	\begin{tabular}{||c l c||} 
	 \hline
	 \emph{Wirtinger Flow} Variant 			 & $\varphi$ 						& loss functions\\ [0.5ex] 
	 \hline\hline
	 Wirtinger Flow 			 			 & $\left|\boldsymbol{z}\right|^2$ 	& quadratic 	\\ 
	 Truncated Wirtinger Flow   			 & $\left|\boldsymbol{z}\right|^2$ 	& quadratic 	\\
	 Incrementally Truncated Wirtinger Flow  & $\left|\boldsymbol{z}\right|^2$  & quadratic 	\\
	 Reshaped Wirtinger Flow 				 & $\left|\boldsymbol{z}\right|$ 	& quadratic 	\\
	 Incrementally Reshaped Wirtinger Flow 	 & $\left|\boldsymbol{z}\right|$ 	& quadratic 	\\
	 Incrementally-Minibatch Wirtinger Flow  & $\left|\boldsymbol{z}\right|$ 	& quadratic 	\\ [1ex] 
	 \hline
	\end{tabular}
	\caption{$\varphi$ and the loss function used in \cite{Candes2014}, \cite{Chen2015}, \cite{Kolte2016}, and \cite{Zhang2016}}
	\label{tab:formulation}
	\end{table}
\section{Difficulties}

The loss function is non-convex. Set $n=1$, $m=2$, $\boldsymbol{x}_1 = \begin{pmatrix}1+i\end{pmatrix}^{1 \times 1}$, 
$\boldsymbol{x}_2 = \begin{pmatrix}-1-i\end{pmatrix}^{1 \times 1}$, $\boldsymbol{A}=\begin{pmatrix}1\\i \end{pmatrix}^{2 \times 1}$, 
$\boldsymbol{y}=\begin{pmatrix}1\\2 \end{pmatrix}^{2 \times 1}$, and $\lambda=1/2$ to build a counterexample. Non-convexity is bad news for 
optimization as it can be seen vividly in \cite{Boyd2004} and \cite{Nocedal2006}. To make the matter worse the loss function is not 
holomorphic( it can be easily seen that Cauchy-Riemann equations\cite{Rudin1987} do not hold) and therefore complex differentiability 
is out of the question\cite{Rudin1987}.

% \begin{equation} \label{prob:mainproblem}
% 	Recover $\boldsymbol{x} \in \mathbb{R}^n/\mathbb{C}^n$ from measurements $y_i$ given by
% 	\begin{flalign}
% 		y_i=\left|\langle \boldsymbol{a}_i,\mathbf{x}\rangle\right|, \quad \text{for }\; i=1,\cdots,m, \label{eq:mainproblem}
% 	\end{flalign}
% 	where $\boldsymbol{a}_i \in \mathbb{R}^n/\mathbb{C}^n$ are random design vectors (known). 
% \end{equation}













\index{Scalar Product}








	\begin{algorithm}
	\caption{Reshaped \emph{Wirtinger Flow} suggested by \cite{Zhang2016}}
		\textbf{Input}: $\boldsymbol{y}=\{y_i\}_{i=1}^m$, $\{\boldsymbol{a}_i\}_{i=1}^m$; \\
		\textbf{Parameters:}  Lower and upper thresholds $\alpha_l,\alpha_u$ for  truncation in initialization, step size $\mu$;\\
		\textbf{Initialization}: Let $\boldsymbol{z}^{(0)}=\lambda_0 \tilde{\boldsymbol{z}}$, where $\lambda_0=\frac{mn}{\sum_{i=1}^m \|\boldsymbol{a}_i\|_1}\cdot \left(\frac{1}{m}\sum_{i=1}^m y_i\right)$ and $\tilde{\boldsymbol{z}}$ is the leading eigenvector of
		\begin{equation}
			\boldsymbol{Y} \coloneqq \frac{1}{m}\sum_{i=1}^m y_i\boldsymbol{a}_i \boldsymbol{a}_i^*\boldsymbol{1}_{\{\alpha_l \lambda_0<y_i< \alpha_u \lambda_0\}}.
		\end{equation}
		\textbf{Update loop}: for $t=0:T-1$ do
		\begin{flalign}
			\boldsymbol{z}^{(t+1)}=\boldsymbol{z}^{(t)}- \frac{\mu}{m}\sum_{i=1}^{m}\left(\boldsymbol{a}_i^*\boldsymbol{z}^{(t)}-y_i\cdot\frac{\boldsymbol{a}_i^*\boldsymbol{z}^{(t)}}{|\boldsymbol{a}_i^*\boldsymbol{z}^{(t)}|} \right) \boldsymbol{a}_i.
		\end{flalign}
		\textbf{Output} $\boldsymbol{z}^{(T)}$.
	\end{algorithm}

	\begin{algorithm}[th] 
		\caption{Incremental Reshaped \emph{Wirtinger Flow} (IRWF) suggested by \cite{Zhang2016}}\label{alg:irwf}
		\textbf{Input}: $\boldsymbol{y}=\{y_i\}_{i=1}^m$, $\{\boldsymbol{a}_i\}_{i=1}^m$; \\
		\textbf{Parameters:}  Lower and upper thresholds $\alpha_l,\alpha_u$ for  truncation in initialization, step size $\mu$;\\
		\textbf{Initialization}: Let $\boldsymbol{z}^{(0)}=\lambda_0 \tilde{\boldsymbol{z}}$, where $\lambda_0=\frac{mn}{\sum_{i=1}^m \|\boldsymbol{a}_i\|_1}\cdot \left(\frac{1}{m}\sum_{i=1}^m y_i\right)$ and $\tilde{\boldsymbol{z}}$ is the leading eigenvector of
		\begin{equation}
			\boldsymbol{Y} \coloneqq \frac{1}{m}\sum_{i=1}^m y_i\boldsymbol{a}_i \boldsymbol{a}_i^*\boldsymbol{1}_{\{\alpha_l \lambda_0<y_i< \alpha_u \lambda_0\}}.
		\end{equation}
		% \textbf{Initialization}: Let $\boldsymbol{z}^{(0)}=\lambda_0 \tilde{\boldsymbol{z}}$, where $\lambda_0=\frac{mn}{\sum_{i=1}^m \|\boldsymbol{a}_i\|_1}\cdot \left(\frac{1}{m}\sum_{i=1}^m y_i\right)$ and $\tilde{\boldsymbol{z}}$ is the leading eigenvector of
		%\begin{equation*}
		%\bY:=\frac{1}{m}\sum_{i=1}^m y_i\ba_i \boldsymbol{a}_i^*\bone_{\{\alpha_l \lambda_0<y_i< \alpha_u \lambda_0\}}.
		%\end{equation*}
		
		 \textbf{Update loop}: for $t=0:T-1$ do\\
		 Choose $i_t$ uniformly at random from $\{1,2,\ldots, m\}$, and let
		  \begin{flalign}
				\boldsymbol{z}^{(t+1)}=\boldsymbol{z}^{(t)}- \mu\left(\boldsymbol{a}_{i_t}^*\boldsymbol{z}^{(t)}-y_{i_t}\cdot\frac{\boldsymbol{a}_{i_t}^*\boldsymbol{z}^{(t)}}{|\boldsymbol{a}_{i_t}^*\boldsymbol{z}^{(t)}|} \right) \boldsymbol{a}_{i_t}, \label{eq:incrementalupdate}
		\end{flalign}
		\textbf{Output} $\boldsymbol{z}^{(T)}$.
		\end{algorithm}



		\begin{algorithm}[th]
			\caption{Mini-batch Incremental Reshaped \emph{Wirtinger Flow} (mini-batch IRWF) suggested by \cite{Zhang2016}}\label{alg:mbirwf}
			
			\textbf{Input}: $\boldsymbol{y}=\{y_i\}_{i=1}^m$, $\{\boldsymbol{a}_i\}_{i=1}^m$; \\
			\textbf{Parameters:}  Lower and upper thresholds $\alpha_l,\alpha_u$ for  truncation in initialization, step size $\mu$;\\
			\textbf{Initialization}: Let $\boldsymbol{z}^{(0)}=\lambda_0 \tilde{\boldsymbol{z}}$, where $\lambda_0=\frac{mn}{\sum_{i=1}^m \|\boldsymbol{a}_i\|_1}\cdot \left(\frac{1}{m}\sum_{i=1}^m y_i\right)$ and $\tilde{\boldsymbol{z}}$ is the leading eigenvector of
			\begin{equation}
				\boldsymbol{Y} \coloneqq \frac{1}{m}\sum_{i=1}^m y_i\boldsymbol{a}_i \boldsymbol{a}_i^*\boldsymbol{1}_{\{\alpha_l \lambda_0<y_i< \alpha_u \lambda_0\}}.
			\end{equation}
			%\textbf{Parameters:}  Lower and upper thresholds $\alpha_l,\alpha_u$ for  truncation in initialization, step size $\mu$;\\
			%\textbf{Initialization}: Let $\bz^{(0)}=\lambda_0 \tilde{\bz}$, where $\lambda_0=\frac{mn}{\sum_{i=1}^m \|\ba_i\|_1}\cdot \left(\frac{1}{m}\sum_{i=1}^m y_i\right)$ and $\tilde{\bz}$ is the leading eigenvector of
			%\begin{equation*}%\label{eq:init_TRWF}
			%\bY:=\frac{1}{m}\sum_{i=1}^m y_i\ba_i \boldsymbol{a}_i^*\bone_{\{\alpha_l \lambda_0<y_i< \alpha_u \lambda_0\}}.
			%\end{equation*}
			
			 \textbf{Update loop}: for $t=0:T-1$ do\\
			 Choose $\Gamma_t$ uniformly at random from the subsets of  $\{1,2,\ldots, m\}$ with cardinality $k$, and let
			 \begin{flalign}
				\boldsymbol{z}^{(t+1)}=\boldsymbol{z}^{(t)}-\mu\cdot\boldsymbol{A}_{\Gamma_t}^* \left(\boldsymbol{A}_{\Gamma_t}\boldsymbol{z}^{(t)}-\boldsymbol{y}_{\Gamma_t}\odot\text{Ph} (\boldsymbol{A}_{\Gamma_t}\boldsymbol{z}^{(t)})\right), \label{eq:mbirwfUpdate}
			\end{flalign}
			where $\boldsymbol{A}_{\Gamma_t}$ is a matrix stacking $\boldsymbol{a}_i^*$ for $i\in \Gamma_t$ as its rows,  $\boldsymbol{y}_{\Gamma_t}$ is a vector stacking $y_i$ for $i\in \Gamma_t$ as its elements, $\odot$ denotes element-wise product, and $\text{Ph}(\boldsymbol{z})$ denotes a phase vector of $\boldsymbol{z}$.% (stacking phase of each element of $\bz$ together).
			
			\textbf{Output} $\boldsymbol{z}^{(T)}$.
			\end{algorithm}


			




\begin{equation*}
	y_k = \left| \sum_{t=0}^{n-1} x[t] e^{-i2\pi\omega_kt} \right|^2 , \qquad \omega_k \in \Omega
  \end{equation*}
  
  \begin{equation*}
	y_k = \left| \sum_{t=0}^{n-1} x[t]\overline{d[t]} e^{-i2\pi\omega_kt} \right|^2 , \qquad \omega_k \in \Omega
  \end{equation*}
  
  \begin{equation*}
	y_k = \left| \sum_{t=0}^{n-1} x[t]\overline{d[t]} e^{-i2\pi\omega_kt} \right|^2 , \qquad \begin{split}
	0 &\leq k \leq n-1\\
	1 &\leq l \leq L
	\end{split}
  \end{equation*}
  
  \begin{equation*}
	\mathbb{E}\left[d\right] = 0, \qquad \mathbb{E}\left[d^2\right] = 0, \qquad\mathbb{E}\left[\left|d\right|^4\right] = 2\mathbb{E}\left[\left|d\right|^2\right]
  \end{equation*}
  
  Ternary Modulation
  \begin{equation*}
	d =
		\begin{cases}
			+1 & \text{with prob.  $1/4$}\\
			0 & \text{with prob.  $1/2$}\\
			-i & \text{with prob.  $1/4$}
		\end{cases}  
  \end{equation*}
  
  
  
  
  
  Octanary Modulation
  \begin{equation*}
	b_1 =
		\begin{cases}
			+1 & \text{with prob.  $1/4$}\\
			-1 & \text{with prob.  $1/4$}\\
			-i & \text{with prob.  $1/4$}\\
			+i & \text{with prob.  $1/4$}\\
  
		\end{cases}  
		\qquad \text{and} \qquad 
	b_2 
		\begin{cases}  
		  +\sqrt{2}/2 & \text{with prob.  $4/5$}\\
		  +\sqrt{3} & \text{with prob.  $1/5$}\\
	  \end{cases}   
  \end{equation*}
  