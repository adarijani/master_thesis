\chapter{Conclusion}

We got almost $4$ orders of magnitude improvement in the relative error both in the \ac{UWF}\index{UWF} and the 
\ac{URWF}\index{URWF} after performing \ac{HP}\cite{Hutter2019}\cite{Akiba2019}\index{\ac{HP}} compared to 
the \ac{WF} and the \ac{RWF} using the same number of iteration which can be seen in 
\cref{fig:uwf_training_07_08_optuna} and \ref{fig:urwf_training_07_08_optuna}. The model is still interpretable thanks 
to the scenarios we considered. The datasets are very small(only $100$ sample points) compared to today's 
\ac{ML}/\ac{DL}\cite{Goodfellow2016}\cite{LeCun2015} counterparts\cite{Krizhevsky2017}\cite{Szegedy2014}. In 
the proposed winning scenarios, different scalars multiplied by a single matrix of the form $\tau_k\boldsymbol{M}$, 
you only need to train $n^2+L$ parameters which tremendously reduces the required number of 
\ac{FLOPS}\cite{Hager2010}\cite{Hennessy2019} and in turn training time to reach satisfactory relative errors both 
on train and test data. To put that into perspective ImageNet\cite{SVLL2021} has around $14$ millions images and a 
relatively new model like GoogleNet\cite{Szegedy2014}, depending on the instance that you use, has millions of 
parameters only to classify an image as one of the $1000$ classes. What got me very interested in taking 
\ac{DU}/\ac{AU}\cite{Monga2019} as my thesis topic was to go for micro-models rather than the general direction 
that is currently trending. If you are reading this, it is my hope that my work could somehow win you over and convince 
you that small does not always mean weak.