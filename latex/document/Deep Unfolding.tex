\chapter{Deep Unfolding}


In 2012 when Alex Krizhevsky's and his supervisors' \dl architecture \cite{Krizhevsky2017} decimated all other competitors in the 
ImageNet \cite{Vision2021} image classification challenge\cite{Vision} 
he ended the harsh winter that the \ml/\ai scientists were experiencing. It was no fluke and nowadays the challenge is always won by \dl 
architectures \cite{Szegedy2014}\cite{He2015}\cite{Simonyan2014} and not carefully handcrafted feature extraction modules. The  might and success of
 \ml/\dl solutions were made possible by powerful parallel 
computers \cite{Meuer} and large data collections gathered since the dawn of Web $2.0$. All and all \dl architectures had great success in a broad
 range of problems ranging from single object image classification \cite{Rawat2017} to real-time multi-object classification and tracking\cite{Luo2021}.
While during the last decade mostly solving the problem at hand was satisfactory, due to a couple of issues(required tremendous raw computational
 power, required large high-quality datasets) which can be troublesome researchers are trying to revisit their approaches 
to tackles the said issues. Training large architecture which can have easily millions of parameters in order to discover the mapping that are fed
 large amount of data can 
only be done on supercomputers which is both costly and time consuming and only a handful of people have access to these kind of resources. With large number of trainable
 parameters comes the need for large clean and high-quality datasets. The ImageNet\cite{Vision2021} image dataset is currently having more than $14$ millions data points which is
  huge. Putting the cost and time aside  
such large architectures lack interpretability and quite the black box mostly as they are usually general convolutions and matrix-vector products combined with general non-linearities. 
This is the exact opposite of iterative algorithms that are drawn from physical models and therefore interpretable at the level that we understand the math and physics behind the model. 
Due to the iterative nature of the signal-image processing technics and similarities \cref{fig:deep_unfolding} to the \dl architectures the attention of the some of the 
researches has been to reconcile these two to benefit from nice feature associated with either of these approaches. When using the model-based 
approach is designing your architecture the number of parameters decrease significantly and the same goes for the need of large datasets. Your initialization of the weights also 
becomes easier and you are more likely to start from a low(given you are looking for a (local)minimum) loss function. If you can write your iteration process in terms of the innate/atomic 
modules of your \dl framework\cite{Google2023}\cite{Chollet2023}\cite{LFMAI2023}, the computation will be done quite fast thanks to years of optimization of the frameworks\cite{Google2023}\cite{Chollet2023}\cite{LFMAI2023} and the near bare-metal programming of the said frameworks 

\section{Multi-Layer Perceptrons}
Let $\boldsymbol{x} \in \mathbb{R}^n,\boldsymbol{y} \in \mathbb{R}^m, \boldsymbol{h}_l \in \mathbb{R}^{k_l}, \boldsymbol{W}^1 \in \mathbb{R}^{k_1 \times n}
,\boldsymbol{W}^{N+1} \in \mathbb{R}^{m \times k_N},\boldsymbol{W}^{l+1} \in \mathbb{R}^{k_l \times k_{l+1}}$ where $m,n,l,k_l \in \mathbb{N}$ and $\varphi$ is a
function from $\mathbb{R}$ to $\mathbb{R}$ with certain properties. 
Consecutive mappings:
\begin{equation}
  \begin{split}
    h_i^{1}   &= \sigma \left( \sum_{j}^{} W_{ij}^{1}x_j + b_i^{1} \right)\\ 
              & \vdots\\
    h_i^{l+1} &= \sigma \left( \sum_{j}^{} W_{ij}^{l+1}h_j^l + b_i^{l+1} \right)\\
              & \vdots\\
    y_i^{}    &= \sigma \left( \sum_{j}^{} W_{ij}^{N+1}h_j^N + b_i^{N+1} \right)
  \end{split}
  \end{equation}
% \begin{equation*}
  % x_i^{l+1} = \sigma \left( \sum_{j}^{} W_{ij}^{l+1}x_j^l + b_i^{l+1} \right)
% \end{equation*}
that take $\boldsymbol{x}$ to $\boldsymbol{y}$ is called a \ac{MLP} architecture.
In the \ml jargon $\boldsymbol{x},\boldsymbol{y},\boldsymbol{h}^j,\boldsymbol{W}^j,\boldsymbol{b}^j,\varphi$ are called \emph{input}, \emph{output}, \emph{hidden variables},
\emph{weight matrices}, \emph{biases}, and \emph{activation function}. \ac{MLP}s were designed to approximate outputs from inputs without knowing the the actual mapping and with observations and measurements. 
It is worth noting that the existence of $\varphi$ is necessary otherwise the pairs $\boldsymbol{W}^j,\boldsymbol{b}^j$ would 
just be an affine transformation and stacking multiple affine transformation would still be an affine transformation and stacking 
such layers would be redundant. Not having an activation function also means the whole mapping would perform poorly when approximating 
complex mappings that do not resemble affine transformation. The schematic involving only the input, output, and hidden variables can be seen 
in \cref{fig:multi_layer_perceptron}.
%%%%%%%%%%%%%%%%%%%%%%%%%%%%%%%%%%%%%%%%%%%%%%%%%%%%%%%%%%%%%
%%%%%%%%%%%%%%%%%% Multi-Layer Perceptron %%%%%%%%%%%%%%%%%%%
%%%%%%%%%%%%%%%%%%%%%%%%%%%%%%%%%%%%%%%%%%%%%%%%%%%%%%%%%%%%%
\begin{figure}
  \centering
  % \captionsetup{justification=centering}
  \resizebox{1.0\textwidth}{!}{  \begin{tikzpicture}[x=2.2cm,y=1.4cm]
    \message{^^JNeural network, shifted}
    \readlist\Nnod{3,4,4,4,3} % array of number of nodes per layer
    \readlist\Nstr{n,p,q,r,m} % array of string number of nodes per layer
    \readlist\Cstr{\strut x,h^{(\prev)},h^{(\prev)},h^{(\prev)},y} % array of coefficient symbol per layer
    \def\yshift{0.5} % shift last node for dots
    
    \message{^^J  Layer}
    \foreachitem \N \in \Nnod{ % loop over layers
      \def\lay{\Ncnt} % alias of index of current layer
      \pgfmathsetmacro\prev{int(\Ncnt-1)} % number of previous layer
      \message{\lay,}
      \foreach \i [evaluate={\c=int(\i==\N); \y=\N/2-\i-\c*\yshift;
                   \index=(\i<\N?int(\i):"\Nstr[\lay]");
                   \x=\lay; \n=\nstyle;}] in {1,...,\N}{ % loop over nodes
        % NODES
        \node[node \n] (N\lay-\i) at (\x,\y) {$\Cstr[\lay]_{\index}$};
        
        % CONNECTIONS
        \ifnum\lay>1 % connect to previous layer
          \foreach \j in {1,...,\Nnod[\prev]}{ % loop over nodes in previous layer
            \draw[connect arrow,white,line width=1.2] (N\prev-\j) -- (N\lay-\i);
            \draw[connect arrow] (N\prev-\j) -- (N\lay-\i);
            %\draw[connect] (N\prev-\j.0) -- (N\lay-\i.180); % connect to left
          }
        \fi % else: nothing to connect first layer
        
      }
      \path (N\lay-\N) --++ (0,1+\yshift) node[midway,scale=1.5] {$\vdots$};
    }
    
    % LABELS
    % \node[above=5,align=center,mygreen!60!black] at (N1-1.90) {input\\[-0.2em]layer};
    % \node[above=2,align=center,myblue!60!black] at (N3-1.90) {hidden layers};
    % \node[above=10,align=center,myred!60!black] at (N\Nnodlen-1.90) {output\\[-0.2em]layer};
    
  \end{tikzpicture}
  
  }
  \caption{$\boldsymbol{x} =\text{input},\boldsymbol{h}^j=\text{hidden variables},\boldsymbol{y}=\text{output}$}
  \label{fig:multi_layer_perceptron}
\end{figure}
%%%%%%%%%%%%%%%%%%%%%%%%%%%%%%%%%%%%%%%%%%%%%%%%%%%%%%%%%%%%%
%%%%%%%%%%%%%%%%%%%%%%%%%%%%%%%%%%%%%%%%%%%%%%%%%%%%%%%%%%%%%
%%%%%%%%%%%%%%%%%%%%%%%%%%%%%%%%%%%%%%%%%%%%%%%%%%%%%%%%%%%%%
\section{Deep Learning}
The advancements in \ac{HPC}\cite{Meuer} in terms of both software\cite{OpenMPCommette}\cite{MPICommittee}\cite{Nvidia} and 
hardware\cite{Patterson2014}\cite{Hennessy2019}\cite{Nvidia}, Mathematical Optimization\cite{Boyd2004}\cite{Nocedal2006}\cite{Sun2019}, 
\ac{AD}\cite{Naumann2011}\cite{Griewank2008} combined with the abundance of data gathered during the Web $2.0$ era gave birth 
to what is currently known as \dl\cite{LeCun2015}\cite{Higham2018}\cite{Berner2021}.  



%%%%%%%%%%%%%%%%%%%%%%%%%%%%%%%%%%%%%%%%%%%%%%%%%%%%%%%%%%%%%
%%%%%%%%%%% Deep Unfolding(Schematic Diagram) %%%%%%%%%%%%%%%
%%%%%%%%%%%%%%%%%%%%%%%%%%%%%%%%%%%%%%%%%%%%%%%%%%%%%%%%%%%%%
\begin{figure}
  \centering
	\captionsetup{justification=centering}
  \resizebox{35em}{10em}{
\begin{tikzpicture}[->,>=stealth',shorten >=1pt,auto,node distance=2cm,
  thick,c node/.style={circle,draw,
  ,minimum size=5mm},c_empt node/.style={circle,draw=none,font=\normalsize
  ,minimum size=5mm},r node/.style={ rectangle,draw,font=\normalsize
  ,minimum height=5mm,minimum width=5mm},a node/.style={single arrow, draw,font=\normalsize, rotate=0}]
  \node[r node] (Input_O) {Input};
  \node[c node] (Iteration)[above of=Input_O] {\normalsize$h(.;\theta)$};
  \node[r node] (Output_O)[above of=Iteration] {Output};
  \node[a node] (Unrolling)[right of=Iteration] {Unrolling}; 
  \node[r node] (Input_U)[right of=Unrolling] {Input}; 
  \node[c node] (iter_1)[right of=Input_U] {\normalsize$h^1(.;\theta^1)$}; 
  \node[c node] (iter_2)[right of=iter_1] {\normalsize$h^2(.;\theta^2)$}; 
  \node[c_empt node] (dotdotdot)[right of=iter_2] {\normalsize$\cdots\cdots$}; 
  \node[r node] (Output_U)[right of=dotdotdot] {Output};
  \node[r node] (interpretable)[below of=iter_2] {Interpretable Layers}; 
  \node[r node] (train)[above of=iter_2] {End to End Training}; 
  \path[every node/.style={font=\sffamily\small,
  		fill=white,inner sep=1pt}]
    (Input_O) edge [ left=30] node[left=1mm] {} (Iteration)    
    (Iteration) edge [loop left] node[left=1mm] {\normalsize$k \rightarrow k+1$} (Iteration)
    (Iteration) edge [ left=30] node[left=1mm] {} (Output_O)
    (Input_U) edge [ left=30] node[above=1mm] {} (iter_1)     
    (iter_1) edge [ left=30] node[above=1mm] {} (iter_2)     
    (iter_2) edge [ left=30] node[above=1mm] {} (dotdotdot)     
    (interpretable) edge [bend left=30] node[above=1mm] {} (iter_1)  
    (interpretable) edge [bend left=30] node[above=1mm] {} (iter_2)
    (train) edge [bend left=-25] node[above=1mm] {} (Input_U)
    (train) edge [bend left=30] node[above=1mm] {} (Output_U)
    ;     
\end{tikzpicture}
}
  \caption{The Schematic Unfolding of an Iterative Algorithm}
  \label{fig:deep_unfolding}
\end{figure}
%%%%%%%%%%%%%%%%%%%%%%%%%%%%%%%%%%%%%%%%%%%%%%%%%%%%%%%%%%%%%
%%%%%%%%%%%%%%%%%%%%%%%%%%%%%%%%%%%%%%%%%%%%%%%%%%%%%%%%%%%%%
%%%%%%%%%%%%%%%%%%%%%%%%%%%%%%%%%%%%%%%%%%%%%%%%%%%%%%%%%%%%%
sdfl