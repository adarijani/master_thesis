\chapter{Introduction}

Physical measurements in settings where light, electrons, and similar entities are involved will result in a phase 
loss\cite{Shechtman2015}\index{phase loss} in the mathematical formulation. The phenomenon is called \acl*{PP}\cite{Shechtman2015}\index{phase problem} 
and the methods developed to tackle the said unpleasantness are called 
\acl*{PR}\index{phase retrieval} methods \cite{Jaganathan2015}\cite{Liu2019}. Due to its presence in a wide spectrum of 
applications\cite{Shechtman2015}\cite{Candes2014} ranging from X-ray crystallography, transmission electron microscopy 
to quantum mechanics, the retrieval methods are highly investigated and coveted. 
One of the contemporary breakthroughs are the \ac{WF}\cite{Candes2014}\index{\ac{WF}} variants\cite{Liu2019} which are nice and relatively easy algorithms 
with small memory footprints equipped with nice guarantees on the solutions. \ac{WF}\index{\ac{WF}} variants \cite{Liu2019} are derived 
from minimizing a certain functional and are of iterative nature. Like most iterative approaches they are certain parameters 
that need to be fixed which greatly influence the convergence rate and stability of the algorithm. While analytical guarantees 
are nice to have, we aim to investigate if it is possible for improvements of the model by altering these parameters using \ac{DU}/\ac{AU}\cite{Monga2019}\index{\du}\index{\au} 
which is an emerging technic from the data-driven world. \ac{DU}\index{\du} is basically unfolding an 
iterative algorithm finite times and putting it into a neural network and training some parameters of the network. \ac{ML}/\ac{DL} studies are often accompanied 
by \ac{HP}\cite{Hutter2019}\cite{Akiba2019} which is why we will be closing our numerical experiments by exactly doing that. 
To explain our train of thought within the numerical experiments we provided a couple of chapters. In the 
second chapter we give a brief introduction to some of the definitions and theorems we need in later 
chapters. \DU and \ac{WF}\index{\ac{WF}} are discussed in chapter three. The results of the numerical experiments that we conducted 
were presented in chapter four. The main two variants that we considered for the \du are the \ac{WF} and the \ac{RWF}. Our 
study was fruitful and we were able to improve the \ac{WF} and the \ac{RWF} by \du which we included in the fifth and the final chapter.