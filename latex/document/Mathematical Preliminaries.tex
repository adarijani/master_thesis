\chapter{Mathematical Preliminaries}
\todo{A few words on what we are supposed to expect in this chapter would be helpful.}
\section{Basic Analysis}
\begin{Def}\label{def:scalar_product_complex_vectors}
    For the complex vectors $\boldsymbol{x},\boldsymbol{y} \in \mathbb{C}^N$ the mapping  $\langle\boldsymbol{.},\boldsymbol{.}\rangle \colon \mathbb{C}^N \times \mathbb{C}^N \to \mathbb{C}$ 
    \begin{equation}
        \langle\boldsymbol{x},\boldsymbol{y}\rangle \coloneqq \sum_{i=1}^{N} x_i \overline{y_i} 
    \end{equation}
    is the scalar product where
    \begin{equation}
        \boldsymbol{x} = \left(x_i\right)_{i=1,\ldots,N}, \quad \boldsymbol{y} = \left(y_i\right)_{i=1,\ldots,N}.
    \end{equation}    
\end{Def}
\begin{Def}\label{def:scalar_product_induced_norm_complex_vectors}
    For the complex vector $\boldsymbol{x} \in \mathbb{C}^N$ the mapping  $\left|\left|\boldsymbol{.}\right|\right| \colon \mathbb{C}^N \to [0,+\infty)$ 
    \begin{equation}
        \left|\left|\boldsymbol{x}\right|\right| \coloneqq  \sqrt{\langle\boldsymbol{x},\boldsymbol{x}\rangle}
    \end{equation}
    is the norm induced by the scalar product for complex vectors where
    \begin{equation}
        \boldsymbol{x} = \left(x_i\right)_{i=1,\ldots,N}.
    \end{equation}    
\end{Def}
\begin{Def}\label{def:scalar_product_induced_norm_complex_matrices}
    For the complex matrices $\boldsymbol{X},\boldsymbol{Y} \in \mathbb{C}^{M \times N}$ the mapping $\langle\boldsymbol{.},\boldsymbol{.}\rangle \colon \mathbb{C}^{M \times N} \times \mathbb{C}^{M \times N} \to \mathbb{C}$ 
    \begin{equation}
        \langle\boldsymbol{X},\boldsymbol{Y}\rangle \coloneqq \sum_{j=1}^{N}\sum_{i=1}^{M} X_{i,j} \overline{Y_{i,j}} 
    \end{equation}
    is the scalar product where
    \begin{equation}
        \boldsymbol{X} = \left(X_{i,j}\right)_{\substack{i=1,\ldots,M\\ j=1,\ldots,N}}, \quad \boldsymbol{Y} = \left(Y_{i,j}\right)_{\substack{i=1,\ldots,M\\ j=1,\ldots,N}}.
    \end{equation}    
\end{Def}

\noindent This scalar product is also called the Frobenius scalar product\index{scalar product!Frobenius}, the Schur scalar product\index{scalar product!Schur} or 
the Hilbert-Schmidt scalar product\index{scalar product!Hilbert-Schmidt}.

\begin{Def}\label{def:p-norm}
    Let $1 \leq p \leq \infty$, and $\boldsymbol{x} \in \mathbb{K}^N$ be a vector, while $\mathbb{K}$ can be either $\mathbb{R}$ or $\mathbb{C}$ then the mapping $\left|\left| \boldsymbol{.} \right|\right|_p \colon \mathbb{K}^N \to [0,+\infty)$
    \begin{equation}
        \begin{split} 
            \left|\left|\boldsymbol{x}\right|\right|_p \coloneqq    
            \begin{cases}
                \left(\sum_{i=1}^{N}\left|x_i\right|^p\right)^{\frac{1}{p}} & \text{for } 1 \leq p < \infty\\
                \underset{{i=1,\dots,N}}{\max} \left|x_i\right| & \text{for } p = \infty,
            \end{cases}
        \end{split}
    \end{equation}
    where $\left|\boldsymbol{\cdot}\right|$ is the absolute value on the $\mathbb{K}$ field,
    is called the \textbf{p-norm} of the vector $\boldsymbol{x}$.
\end{Def}
\begin{Thm}\label{theorem:euler_formula}
    Let $\alpha,\beta,\theta, \in \mathbb{R}$ and $\mathrm{i}$ the imaginary unit associated with the complex numbers in $\mathbb{C}$ 
    then we would be having:
	\begin{equation}
		\mathrm{e}^{\mathrm{i}\theta} = \cos \theta +  \mathrm{i}\sin \theta \qquad  \mathrm{e}^{\mathrm{i}(\alpha+\beta)} = \mathrm{e}^{\mathrm{i}\alpha}\mathrm{e}^{\mathrm{i}\beta}\index{Euler formula}\index{Function!$\exp$}\index{Function!$\sin$}\index{Function!$\cos$}
	\end{equation}
    where $\sin$\index{Function!$\sin$} and $\cos$\index{Function!$\cos$} are the usual trigonometric functions.
	\end{Thm}
	\begin{Rem}
	It is not so obvious why multiplication of two infinite series($\mathrm{e}^{\mathrm{i}\alpha}$ and $\mathrm{e}^{\mathrm{i}\beta}$) would converge, let alone to a specific function namely  $\mathrm{e}^{\mathrm{i}(\alpha+\beta)}$.
    It is recommended for the curious reader to check out \emph{The Exponential and Logarithmic Functions} and \emph{The Trigonometric Functions} in Chapter 8 of \cite{Rudin1976}. 
	\end{Rem}
	\begin{Lem}
		Let $a$, $b$, $\theta$, and $\varphi$ $\in \mathbb{R}$ and $\sin$\index{Function!$\sin$}, $\cos$\index{Function!$\cos$} 
        the usual trigonometric functions then:
		\begin{equation}
			a\sin \theta+b\cos \theta  = \sqrt{a^2+b^2}\sin(\theta + \varphi)
		\end{equation}
		such that:
		\begin{equation}
			\sin\varphi=\frac{b}{\sqrt{a^2+b^2}}, \qquad \cos\varphi = \frac{a}{\sqrt{a^2+b^2}}
		\end{equation}
		\end{Lem}


\section{Discrete Fourier Transform}
The \ac{FT} and its discrete counterpart will play an important role in one of our motivating examples\cref{sec:diffracted_imaging} 
for the \ac{WF} and here we give a fast refresher on the basic properties of the \ac{DFT} and the \ac{IDFT}.

\begin{Def}[1D \ac{DFT} and 1D \ac{IDFT}]\label{def:1ddft_1didft}
    The 1D \ac{DFT}  of the vector $\boldsymbol{x} \in \mathbb{C}^{N}$ is denoted by the vector $\hat {\boldsymbol{x}} \in \mathbb{C}^{N}$ and is defined by:
    \begin{equation}\label{eq:1ddft}
        {\hat x}_k \coloneqq \frac{1}{N}\sum_{n=0}^{N-1} x_n\exp\left({\frac{-2\pi ink}{N}}\right)\index{\ac{FT}!$1$-$d$ \ac{DFT}}.
    \end{equation}
    The 1D \ac{IDFT}  of the vector $\boldsymbol{x} \in \mathbb{C}^{N}$ is denoted by the vector $\check {\boldsymbol{x}} \in \mathbb{C}^{N}$ and is defined by:
    \begin{equation}\label{eq:1didft}
        \check {x}_k \coloneqq \sum_{k=0}^{N-1}x_{n}\exp\left({\frac{2\pi ink}{N}}\right)\index{\ac{FT}!$1$-$d$ \ac{IDFT}}.
    \end{equation}    
\end{Def}

\begin{Rem}
As it is evident from the formula the 1D \ac{DFT} is a linear transformation therefore there is a corresponding matrix 
$\boldsymbol{W}_N \in \mathbb{C}^{N\times N}$ that takes $\boldsymbol{x} \in \mathbb{C}^N$ to $\boldsymbol{y} \in \mathbb{C}^N$. 
The matrix $\boldsymbol{W}_N$ can be shown as:

\begin{equation}
    \boldsymbol{W}_N \coloneqq 
    \begin{bmatrix}
        1     & 1                & 1                   & 1                   & \cdots & 1                      \\
        1     & \omega_{N}^{}    & \omega_{N}^{2}      & \omega_{N}^{3}      & \cdots & \omega_{N}^{N-1}       \\
        1     & \omega_{N}^{2}   & \omega_{N}^{4}      & \omega_{N}^{6}      & \cdots & \omega_{N}^{2(N-1)}    \\
        1     & \omega_{N}^{3}   & \omega_{N}^{6}      & \omega_{N}^{9}      & \cdots & \omega_{N}^{3(N-1)}    \\
        \cdot & \cdot            & \cdot               & \cdot               & \cdots & \cdot                  \\ 
        \cdot & \cdot            & \cdot               & \cdot               & \cdots & \cdot                  \\ 
        1     & \omega_{N}^{N-1} & \omega_{N}^{2(N-1)} & \omega_{N}^{3(N-1)} & \cdots & \omega_{N}^{(N-1)(N-1)}
        \end{bmatrix}
\end{equation}
where $\omega_N = e^{-2\pi i/N}$ and $\omega_N^{mn} = e^{-2\pi imn/N}$. Considering again the $\boldsymbol{x}$
and the $\hat {\boldsymbol{x}}$ vectors we can rewrite the 1D \ac{DFT}:
\begin{equation}
    \hat {\boldsymbol{x}} = \frac{1}{N}\boldsymbol{W}_N\boldsymbol{x}
\end{equation}
 The matrix is a dense matrix\cite{Frazier1999} and due to computational 
efficiency\footnote{both in terms of number of FLOPS and memory footprint}\cite{Frazier1999} is almost never computed directly, 
however taking a closer look at the basis vectors would shed some light on the nature of the said transform and is a 
time well spent.
\end{Rem}
\begin{Prop}\label{Prop:1ddft_vectors_orthononality}
    The vectors 
    \begin{equation}\label{eq:1ddft_vectors}
        \boldsymbol{g}^n = \left(\exp\left({\frac{-2\pi ink}{N}}\right)\right)_{k=0,\ldots,N-1}
    \end{equation}
    are orthogonal to each other with respect to the scalar product in \cref{def:scalar_product_induced_norm_complex_vectors} and 
    each of them has a norm, induced by the scalar product, of $\sqrt{N}$ which can be formulated as: 
    \begin{equation}
        \langle\boldsymbol{g}^n,\boldsymbol{g}^{n'}\rangle= N \delta_{n,n'} \qquad n \land n'\in \left\{0,\ldots,N-1\right\}
    \end{equation}
\end{Prop}

\begin{Proof}\label{proof:1ddft_vectors_orthononality}
    Recall that
    \begin{equation}\label{eq:geometric_sum_formula}
        \sum_{i=0}^{N-1} q^i = \frac{1-q^N}{1-q} \quad \forall q \in \mathbb{C}\backslash\left\{1\right\}
    \end{equation}
    which is known as the geometric sum formula. 
    \begin{equation}
    \begin{split} 
        \langle\boldsymbol{g}^n,\boldsymbol{g}^{n'}\rangle &= \sum_{k=0}^{N-1} \exp\left({\frac{-2\pi ink}{N}}\right)\overline{\exp\left({\frac{-2\pi in'k}{N}}\right)}
        = \sum_{k=0}^{N-1} \exp\left({\frac{-2\pi ink}{N}}\right)\exp\left({\frac{+2\pi in'k}{N}}\right)\\
        &= \sum_{k=0}^{N-1} \exp\left({\frac{-2\pi i(n'-n)k}{N}}\right)\\
        &=
        \begin{cases}
            N & \text{when $n = n'$}\text{(trivial)},\\
            0 & \text{when $n\neq n'$}\text{(Eq. \ref{eq:geometric_sum_formula} for $q = \exp\left({\frac{-2\pi i(n'-n)}{N}}\right)$)}.
        \end{cases}
    \end{split}
\end{equation}
\end{Proof}

\begin{Prop}
    The vectors in \cref{Prop:1ddft_vectors_orthononality} form a basis.
\end{Prop}
\begin{Def}[2D \ac{DFT} and 2D \ac{IDFT}]\label{def:2ddft_2didft}
    The 2D \ac{DFT} of the matrix $\boldsymbol{X} \in \mathbb{C}^{N \times M}$ is denoted by 
    the matrix $\hat {\boldsymbol{X}} \in \mathbb{C}^{N \times M}$ and is defined by:
    \begin{equation}\label{eq:2ddft}
        {\hat X}_{k,l} \coloneqq \frac{1}{MN}\sum_{m=0}^{M-1}\sum_{n=0}^{N-1} X_{n,m}\exp\left({\frac{-2\pi ink}{N}}\right)\exp\left({\frac{-2\pi iml}{M}}\right).
    \end{equation}
    The 2D \ac{IDFT} of the matrix $\boldsymbol{X} \in \mathbb{C}^{N \times M}$ is denoted by 
    the matrix $\check {\boldsymbol{X}} \in \mathbb{C}^{N \times M}$ and is defined by:
    \begin{equation}\label{eq:2didft}
        \check X_{n,m} \coloneqq \sum_{k=0}^{N-1}\sum_{l=0}^{M-1}X_{k,l}\exp\left({\frac{2\pi ink}{N}}\right)\exp\left({\frac{2\pi iml}{M}}\right)
    \end{equation}    
\end{Def}

\begin{Rem}
Like the 1D \ac{DFT} it can be seen from the formula that the 2D \ac{DFT} is a linear transformation. 
The interesting thing in the 2D case that it is basically 2 consecutive 1D \ac{DFT} on the rows and columns. 
The order of the 1D \ac{DFT}s do not matter and you can take the 1D \ac{DFT} on the rows and then 1D 
\ac{DFT} on the columns or vice versa. The denseness and computational efficiency still hold which is why 
we again take a closer look at the properties of the matrices that were introduced in \cref{def:2ddft_2didft} in the hope gaining
some insight into the 2D {DFT}.
\end{Rem}

\begin{Prop}\label{Prop:2ddft_matrices_orthogonality}
    The matrices
    \begin{equation}\label{eq:2ddft_matrices}
        \boldsymbol{g}^{n,m} = \left(\exp\left({\frac{-2\pi ink}{N}}\right)\exp\left({\frac{-2\pi iml}{M}}\right)\right)_{\substack{k=0,\ldots,N-1\\l=0,\ldots,M-1}}
    \end{equation}
    are orthogonal to each other with respect to the scalar product in \cref{def:scalar_product_induced_norm_complex_matrices} and 
    each of them has a norm, induced by the scalar product, of $\sqrt{MN}$ which can be formulated as:
    \begin{equation}
        \langle\boldsymbol{g}^{n,m},\boldsymbol{g}^{n',m'}\rangle= MN \delta_{n,n'}\delta_{m,m'} \qquad (n \land n' \in \left\{0,\ldots,N-1\right\}) \land (m \land m' \in \left\{0,\ldots,M-1\right\})
    \end{equation}
\end{Prop}

\begin{Proof}
    \begin{equation}
        \begin{split}  
            \langle\boldsymbol{g}^{n,m},\boldsymbol{g}^{n',m'}\rangle &= \sum_{l=0}^{M-1}\sum_{k=0}^{N-1} \exp\left({\frac{-2\pi ink}{N}}\right)\exp\left({\frac{-2\pi iml}{M}}\right)\overline{\exp\left({\frac{-2\pi in'k}{N}}\right)\exp\left({\frac{-2\pi im'l}{M}}\right)}\\
            &= \sum_{l=0}^{M-1}\sum_{k=0}^{N-1} \exp\left({\frac{-2\pi ink}{N}}\right)\exp\left({\frac{-2\pi iml}{M}}\right)\exp\left({\frac{+2\pi in'k}{N}}\right)\exp\left({\frac{+2\pi im'l}{M}}\right)\\
            &= \sum_{l=0}^{M-1}\sum_{k=0}^{N-1} \exp\left({\frac{-2\pi i(n'-n)k}{N}}\right)\exp\left({\frac{-2\pi i(m'-m)l}{M}}\right)\\
            &= \sum_{k=0}^{N-1} \exp\left({\frac{-2\pi i(n'-n)k}{N}}\right)\sum_{l=0}^{M-1} \exp\left({\frac{-2\pi i(m'-m)k}{M}}\right)\\
            &=
            \begin{cases}
                MN & \text{$n = n' \wedge m=m'$}\text{(trivial)},\\
                0 & \text{$\neg(n = n' \wedge m=m')$}\text{(just like the 1D case in \cref{proof:1ddft_vectors_orthononality})}.
            \end{cases}  
        \end{split}
    \end{equation}    
\end{Proof}
\begin{Prop}
    The matrices in \cref{Prop:2ddft_matrices_orthogonality} form a basis.
\end{Prop}
\begin{Rem}
Due to its wide range of applications the \ac{DFT} the \ac{IDFT} are ranked third in \cite{KARBBCJGPHKKDPWLPJSSWWKY2006}
\footnote{This is a very important and highly cited paper due to its importance in the \ac{HPC}. Important numerical algorithms
are referred to as \emph{dwarves} humorously. Dense Linear Algebra, Sparse Linear Algebra, 
Discrete Fourier Transform are the first, the second and the third dwarf in the descending order of importance. 
It is not dramatizing to say that the first three \emph{dwarves} stature in the numerical world 
is the same as the three great Olympians(Zeus, Poseidon, Hades) in the Greek mythology.}, the first and the 
second being dense linear algebra and sparse linear algebra. As you can imagine implementing your own library is not recommended due to the notoriously detailed implementation issue associated with the \ac{DFT} and the \ac{IDFT}. 
The practitioner that need the \ac{FT} probably should look into specifically designed packages like \cite{KARBBCJGPHKKDPWLPJSSWWKY2006} orthogonality \cite{VGOHRCBPWBWBWMMNJKLCPFMVLPCHQHARPM2020}.
\end{Rem}
\begin{Prop}
    The \ac{DFT} in both 1D and 2D is\footnote{Please note that the \begin{equation}
        \langle\boldsymbol{g}^n,\boldsymbol{g}^{n'}\rangle= N \delta_{n,n'} \qquad n \land n'\in \left\{0,\ldots,N-1\right\}
    \end{equation} and     \begin{equation}
        \langle\boldsymbol{g}^{n,m},\boldsymbol{g}^{n',m'}\rangle= MN \delta_{n,n'}\delta_{m,m'} \qquad (n \land n' \in \left\{0,\ldots,N-1\right\}) \land (m \land m' \in \left\{0,\ldots,M-1\right\})
    \end{equation} which is why it is possible to normalize the formulas to get a unitary transformation. We are not going to 
    do that to be consistent with the way the \ac{DFT} and the \ac{IDFT} are introduced in most of the literature.} an orthogonal transformation   
\end{Prop}

\begin{Rem}
    The \ac{DFT} being orthogonal is a nice property and we are going to put that to use in \cref{sec:diffracted_imaging}. 
\end{Rem}



























