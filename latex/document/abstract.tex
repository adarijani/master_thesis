\chapter*{Abstract}
\addcontentsline{toc}{chapter}{Abstract}  



\acl{SDL} \ac{ISTA} \ac{LISTA} \ac{FISTA} \ac{Ada-LISTA}

\ac{DU}/\ac{AU}
Physical measurements in settings where light, electrons, and similar existences are involved will result in a phase 
loss\cite{Shechtman2015} in the mathematical formulation. The phenomenon is called \acl*{PP}\cite{Shechtman2015} 
and the methods developed to tackle the said unpleasantness are called 
\acl*{PR}\cite{Jaganathan2015}\cite{Liu2019} methods. Due to its presence in a wide spectrum of 
applications\cite{Shechtman2015}\cite{Candes2014} ranging from X-ray crystallography, transmission electron microscopy 
to quantum mechanics, the retrieval methods are highly investigated and coveted\cite{Jaganathan2015}\cite{Liu2019}. 
One of the contemporary breakthroughs are the \ac{WF}\cite{Candes2014}\cite{Liu2019} variants which are nice and relatively easy algorithms 
with small memory footprints equipped with nice guarantees on the solutions. \ac{WF}\cite{Liu2019} variants are derived from minimizing a 
certain functional and are of iterative nature. Like most iterative approaches they are certain parameters that need to be fixed which 
greatly influence the convergence rate and stability of the algorithm. While analytical guarantees are nice to have, we aim 
to investigate if it is possible for improvements of the model by altering these parameters using \ac{DU}/\ac{AU}\cite{Monga2019} 
which is an emerging technic from the data-driven world. \ac{DU}/\ac{AU} is basically unfolding/unrolling an 
iterative algorithm finite times and putting it into a neural network to be trained. \ac{ML}\ac{DL} studies are often accompanied 
by \ac{HP} which is why we close the current work by exactly doing that.   
\endinput